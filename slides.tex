\documentclass[12pt]{beamer}
\usetheme{Rochester}
\usepackage[utf8]{inputenc}
\usepackage{amsmath}
\usepackage{amsfonts}
\usepackage{amssymb}
\usepackage{graphicx}
\usepackage{makecmds}

\author{Kyle Treleaven}
\title{State-space, Equilibria, and Stability in the sense of Lyapunov}
%\setbeamercovered{transparent} 
%\setbeamertemplate{navigation symbols}{} 
%\logo{} 
%\institute{} 
%\date{} 
%\subject{} 

\usefonttheme[onlymath]{serif}

% typesetting
\makecommand{\vitem}{\vfill\item}

% general math
\makecommand{\reals}{{\mathbb R}}
\makecommand{\timederiv}{\frac{d}{dt}}
\makecommand{\onehalf}{\frac{1}{2}}

% specific math
\makecommand{\zerovec}{{\bf 0}}

\makecommand{\vecu}{{\bf u}}
\makecommand{\vecx}{{\bf x}}
\makecommand{\matA}{{\bf A}}
\makecommand{\matC}{{\bf C}}
\makecommand{\vectheta}{{\boldsymbol\theta}}

\begin{document}

\begin{frame}
\titlepage

\end{frame}

%\begin{frame}
%\tableofcontents
%\end{frame}


\begin{frame}
\frametitle{State-space Philosophy}

(put the Wikipedia quote?)

\begin{itemize}
\vitem We begin observing a signal, or ``trajectory'',
\[
y(t) \in \reals.
\]

\vitem We call the present moment $t=0$, and are interested in $y(t)$ over $t \geq 0$ (future).

%\vitem Many trajectories are possible, but there are ``dynamical constraints''.

\vitem Goal is to find another signal $\vecx(t) \in \reals^d$ (perhaps more complicated), and functions $f : \reals^d \to \reals^d$ and $g : \reals^d \to \reals$, so that:

\begin{enumerate}
\vitem The system
\[
\begin{cases}
	\dot x = f(x) \\
	x(0) = x_0
\end{cases}
\]
has a unique solution over $t \geq 0$;

In other words, knowing the state at a particular moment in time is sufficient to simulate the entire trajectory.

\vitem $y(t) = g( x(t) )$ over $t \geq 0$;

In other words, applying $g$ to the state-trajectory recovers the desired signal.

\vitem Every possible trajectory can be recovered using some \emph{initial condition} $x_0$.

\end{enumerate}


\end{itemize}

\vfill\null
\end{frame}



\begin{frame}
\frametitle{Review of Last Time: Second-order State-space Model}

(paste a spring-mass-damper image here)

\begin{block}{Second-order, Linear Dynamical System}
$u(t)$ describes the position of a mass over time.

Known to obey the law:
$m \ddot u = -k u - c \dot u
	\implies\qquad
	\ddot u = -(k/m) u - (c/m) \dot u$,
	
where $\dot u = \timederiv u(t)$ (velocity), $\ddot u = \frac{d^2}{dt^2} u(t)$ (acceleration).
\end{block}

\end{frame}






\begin{frame}
\begin{block}{State-space Model}
Write
$\vecu(t) = \left( u(t), \dot u(t) \right)^T$;
also written, 
$\vecu = \begin{pmatrix} u \\ \dot u \end{pmatrix}$.

(Element-wise derivative)

$\dot\vecu
	\doteq \timederiv \vecu(t)
	= ( \timederiv u(t), \timederiv \dot u(t) )^T
	= \begin{pmatrix}
		\dot u \\
		\ddot u
	\end{pmatrix}$.

\end{block}

\begin{block}{SMD Model}
$\dot \vecu
	= \begin{pmatrix} \dot u \\ \ddot u \end{pmatrix}
	= \begin{pmatrix}
		\dot u \\
		-(k/m) u - (c/m) \dot u
		\end{pmatrix}$
		
$\dot \vecu
	= \begin{pmatrix} \dot u \\ \ddot u \end{pmatrix}
	= \begin{pmatrix}
		\dot u \\
		-(k/m) u - (c/m) \dot u
		\end{pmatrix}
	= \begin{pmatrix} 0 & 1 \\ -\frac{k}{m} & -\frac{c}{m} \end{pmatrix}
	\vecu$.
	
\end{block}

\end{frame}


\begin{frame}

\begin{itemize}

\vitem
$\matA := \begin{pmatrix} 0 & 1 \\ -\frac{k}{m} & -\frac{c}{m} \end{pmatrix}$

\vitem
Then $\dot\vecu$ is given by $f(\vecu) = \matA \vecu$

\vitem $\matC := \begin{pmatrix} 1 & 0 \end{pmatrix}$

\vitem
Then $u(t)$ is given by
$g(\vecu) = \matC \vecu$.

\end{itemize}
\vfill\null

\end{frame}


\begin{frame}
\frametitle{Solution}

\begin{itemize}

\vitem Admits a unique state-space trajectory determined by initial conditions $\vecu(0) = \vecu_0$
\[
  \vecu(t) = \vecu_0 e^{t \matA }
  \qquad t \geq 0.
\]

\vitem Particle trajectory is given by $u(t) = \matC \vecu(t)$ for $t \geq 0$.

\end{itemize}
\vfill\null

\end{frame}




\begin{frame}
\frametitle{Equilibria}

\begin{itemize}

\vitem \emph{Equilibria} are conditions such that the system ``stays put''.

\vitem Given a state-space $X$, the equilibria are 
\[
\left\{
	\vecx : \ f(\vecx) = \zerovec \implies \dot \vecx = \zerovec
\right\}
\]

\vitem In our example:
\[
f(\vecx) = \zerovec \implies
\begin{cases}
	\dot u = 0	\\
	\ddot u = -(k/m) u - (c/m) \dot u = 0
\end{cases}
\]

\vitem If $k > 0$, equilibrium of $u = 0$ and $\dot u = 0$ is \emph{unique}.

\vitem If $k = 0$, all $u$ are equilibria when $\dot u = 0$.

\end{itemize}

\end{frame}



\begin{frame}

Put ``state-space'' locus plots of equilibria in the two cases.

\end{frame}



\begin{frame}

(Put a picture)

Pendulum trajectories satisfy

$\ddot\theta + \frac{g}{l} \sin \theta = 0$.

Use angle and angular velocity as a state-space.

$\vectheta = (\theta, \dot \theta)^T$

$f(\vectheta)
	= \begin{pmatrix}
		\dot\theta \\
		-\frac{g}{l} \sin\theta
	\end{pmatrix}$.
	
$g > 0$ (thanks, universe).

\[
\begin{cases}
\dot\theta = 0 \\
\sin \theta = 0
\end{cases}
\]

Two distinct equilibria:

$\dot\theta = 0$ and 1) $\theta=0$, 2) $\theta=\pi$.

(put pictures)

Remark: There is an ``instability'' theory, but here only focus on proving an equilbrium \emph{is} stable.

\end{frame}



\begin{frame}
\frametitle{Stability}

In general, stability is a property of EQUILIBRIA, \emph{not} systems!

Stability of ``systems'' is a colloquialism which arose because of the behavior of linear systems;
they tend to have unique equilibrium at origin.

But, for example, it doesn't make any sense to ask is the pendulum stable.

\begin{block}{Types of Stability}
Lyapunov,
Asymptotic
Exponential
\end{block}

\end{frame}




\begin{frame}
\frametitle{Lyapunov's Second (``Direct'') Method}
\end{frame}



\begin{frame}
\frametitle{Pendulum Example}

\[
E = mgh + \onehalf m v^2.
\]

$h = l(1-\cos\theta)$

$v = (l\dot\theta)^2$

\[
E
%ml \, g(1-\cos\theta) + \onehalf ml^2 \, \dot\theta^2
	= ml \left( g \, (1-\cos\theta) + \onehalf l \, \dot\theta^2 \right)
\]

\[
E' = g \, (1-\cos\theta) + \onehalf l \, \dot\theta
\]

(plot the surface, show positive definite!)

\end{frame}



\begin{frame}
\frametitle{Pendulum Example}
\[
\frac{\partial}{\partial\theta} E' = g \sin \theta
\]

\[
\frac{\partial}{\partial\dot\theta} E' = l \dot\theta
\]

\[
\dot E' =
	\frac{\partial}{\partial\theta} E' 
		\times f_{\theta}(\vectheta)
	+ \frac{\partial}{\partial\dot\theta} E'
		\times f_{\dot\theta}(\vectheta)
\]


\[
\dot E' =
	\left( g \sin\theta \times \dot\theta \right)
	+ \left( l \dot\theta \times (-g/l) \sin \theta \right) = 0
\]

(This is something we expected, being familiar with physics and conservation of total energy.)

Proof that the bottom equilibrium is \emph{marginally} stable i.s.L.

\end{frame}





\begin{frame}
\frametitle{Stability for LTI Systems (Continuous)}
$\dot x = Ax$

The notation $P > 0$ means that matrix P is positive definite.
Given any $Q > 0$, there exists a unique $P > 0$ satisfying 
$A^T P + P A + Q = 0$ if and only if the origin is the globally asymptotically stable equilibrium of the linear system $\dot x = A x$.

If and only if $\lambda \in eig(A) => Re(\lambda) < 0$

The quadratic function $V(z) = z^T P z$ is a Lyapunov function that can verify stability.

\end{frame}



\begin{frame}
\frametitle{Pathologies}

Examples: globally attractive equilibrium, not stable, even i.s.L. (weakest sense)

also, "radial unboundedness" ("escape")

\end{frame}








\begin{frame}
\frametitle{Remarks}

There is an instability theory

Lyapunov's Indirect Method (Linearization)

Similar theories for discrete systems (e.g., computer programs)

\end{frame}




\begin{frame}{Review of Second-order State-space Model (from last time)}

\begin{equation}
V = b \left( y - x^{2}\right)
	\left(
		-\onehalf a + \frac{b}{2c} \left(k + m\right) \left(- x^{2}{\left (t \right )} + y{\left (t \right )}\right) + 0.5 x{\left (t \right )}\right)
		+ \left(a - x{\left (t \right )}\right) \left(- 0.5 b \left(- x^{2}{\left (t \right )} + y{\left (t \right )}\right) + \frac{0.5}{c k} \left(a - x{\left (t \right )}\right) \left(c^{2} + k m + m^{2}\right)\right)
\end{equation}

\begin{equation}
\dot V = -b^2 \left( y - x^2 \right) - \left( a - x \right)^2
\end{equation}

\end{frame}













\begin{frame}\end{frame}
\begin{frame}\end{frame}
\begin{frame}\end{frame}
\begin{frame}\end{frame}





\end{document}