%\documentclass[10pt,a4paper,twocolumn]{article}
\documentclass[10pt,a4paper]{article}
\usepackage[latin1]{inputenc}
\usepackage{amsmath}
\usepackage{amsfonts}
\usepackage{amssymb}
\usepackage{graphicx}

\usepackage{makecmds}

\makecommand{\ie}{i.e.}
\makecommand{\eg}{e.g.}


\title{A model of productivity at the interface with science and engineering}
\author{Kyle~Treleaven,~Ph.D.}

\begin{document}

\maketitle

\section{Introduction}

I have been a scientist, and I have been an engineer.

\section{Background}

\section{Problem Statement}

\subsection{Terminology}

\makecommand{\naturals}{{\mathbb N}}

\section{A model of productivity in innovative fields}

\makecommand{\statespace}{X}
\makecommand{\systemfunc}{f}

\makecommand{\actclass}{A}
\makecommand{\actclasses}{{\mathcal A}}
\makecommand{\resclass}{R}
\makecommand{\resclasses}{{\mathcal R}}

\makecommand{\rhist}{h}
\makecommand{\timevar}{\tau}

The proposed mathematical model of productivity includes two types of objects, namely, \emph{activity classes}, and \emph{asset, or resource, classes}.
A productivity \emph{dynamics} is the tuple of a set of activity classes $\actclasses$ and a set of resource classes $\resclasses$, \ie~$(\actclasses,\resclasses)$.
An activity class $\actclass \in \actclasses$ is characterized by a set of \emph{activations}.
We propose a model for activations meant to capture the dynamics of utilizing assets over time to generate value.
To represent the unique dynamics of productivity in innovative fields, the model also includes the effects of devoting resources to the generation and refinement of \emph{value-creating activities}, as well as the discount in value of such activities---techniques or technologies---as they age.
The effects of activation (of an activity) can include the \emph{generation} of new activities, or the \emph{transformation} of activities from one class to another.
Resource classes do not require characterization beyond identity.


\section{Event-based model}

An activation is the pair of a \emph{resource commitment} with an \emph{activation effect}.
A resource commitment is the tuple of a histogram $\rhist$ over resource classes and a commitment duration $\timevar$, to wit $(\rhist, \tau)$.


\section{Discrete-time model}

\section{A continuous model}

\subsection{Simple continuous productivity model}

\makecommand{\genEffort}{x}
\makecommand{\opprate}{\lambda}
\makecommand{\oppvalue}{v}
\makecommand{\oppdecay}{\beta}

\makecommand{\useEffort}{y}
\makecommand{\horizon}{\tau}

The control in this problem is division of a single unit of effort.
An effort $\genEffort$ generates new \emph{opportunities} at a continuous rate of $\opprate(\genEffort)$,
often simply $\opprate\genEffort$.
A unit of opportunity initially has value $\oppvalue_0$, but
that value decays exponentially over time, such that
a unit of opportunity age $t$ has value $\oppvalue_0 e^{-\oppdecay t}$
for $\oppdecay > 0$.
%
The remaining effort $\useEffort$ is spent exploiting existing opportunities.
Generally, the amount of opportunity exploited per unit time is an increasing, often continuous and perhaps linear function of the effort.
The most value is derived by spending effort on the freshest (least decayed) opportunities,
therefore the reward rate can be determined in terms of a ``horizon'' function  $\horizon(\genEffort,\useEffort)$, specifying the maximum age of an active opportunity.
In the doubly-linear case (i.e., production and exploitation both linear in effort),
we have simply $\horizon(\genEffort,\useEffort) = \horizon \useEffort / \genEffort$.
This leads ultimately to the reward function
%
\begin{equation}
R = \int_{t=0}^{\horizon(\genEffort,\useEffort)}
	\opprate(\genEffort) \, \oppvalue_0 e^{-\oppdecay t} \, dt,
\end{equation}
%
or often simply
%
%
\begin{equation}
R = \int_{t=0}^{\horizon \useEffort / \genEffort}
	\opprate \, \genEffort \, \oppvalue_0 e^{-\oppdecay t} \, dt
\end{equation}
%
or
\begin{equation}
R = \frac{\opprate \oppvalue_0}{\oppdecay} \genEffort
\left[
	1 - \left( e^{-\oppdecay \horizon} \right)^{\useEffort / \genEffort }
\right].
\end{equation}

Note that the factor $\oppdecay \horizon$ determines the optimal distribution of a unit of effort.


















\end{document}