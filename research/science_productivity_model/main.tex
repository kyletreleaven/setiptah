%\documentclass[10pt,a4paper,twocolumn]{article}
\documentclass[10pt,a4paper]{article}
\usepackage[latin1]{inputenc}
\usepackage{amsmath}
\usepackage{amsfonts}
\usepackage{amssymb}
\usepackage{graphicx}

\usepackage{makecmds}

\makecommand{\ie}{i.e.}
\makecommand{\eg}{e.g.}


\title{A model of productivity at the interface with science and engineering}
\author{Kyle~Treleaven,~Ph.D.}

\begin{document}

\maketitle

\section{Introduction}

I have been a scientist, and I have been an engineer.

\section{Background}

\section{Problem Statement}

\subsection{Terminology}

\makecommand{\naturals}{{\mathbb N}}

\section{A model of productivity in innovative fields}

\makecommand{\statespace}{X}
\makecommand{\systemfunc}{f}

\makecommand{\actclass}{A}
\makecommand{\actclasses}{{\mathcal A}}
\makecommand{\resclass}{R}
\makecommand{\resclasses}{{\mathcal R}}

\makecommand{\rhist}{h}
\makecommand{\timevar}{\tau}

The proposed mathematical model of productivity includes two types of objects, namely, \emph{activity classes}, and \emph{asset, or resource, classes}.
A productivity \emph{dynamics} is the tuple of a set of activity classes $\actclasses$ and a set of resource classes $\resclasses$, \ie~$(\actclasses,\resclasses)$.
An activity class $\actclass \in \actclasses$ is characterized by a set of \emph{activations}.
We propose a model for activations meant to capture the dynamics of utilizing assets over time to generate value.
To represent the unique dynamics of productivity in innovative fields, the model also includes the effects of devoting resources to the generation and refinement of \emph{value-creating activities}, as well as the discount in value of such activities---techniques or technologies---as they age.
The effects of activation (of an activity) can include the \emph{generation} of new activities, or the \emph{transformation} of activities from one class to another.
Resource classes do not require characterization beyond identity.


\section{Preliminaries: A dynamic (state-space) model}

\makecommand{\statespace}{{\mathcal X}}
\makecommand{\statevar}{x}
\makecommand{\statevec}{{\rm \statevar}}
\makecommand{\controlvar}{u}
\makecommand{\controlvec}{{\rm \controlvar}}
\makecommand{\dynfunc}{f}
\makecommand{\valuevar}{V}
\makecommand{\valuerate}{g}

\makecommand{\classvar}{C}

One modeling approach is to employ a pool of activities,
growing in turn due to the engagement in activity-generating activations.
%
There may be several \emph{types} of activities, distinct in the way they evolve over time (their \emph{dynamics}).
For an activity of type $t$, we can denote the activity state space as $\statespace_t$, and as usual, dynamics are
$\statevec_+ = \dynfunc(\statevec, \controlvec)$.
Suppose also that the activity generates value at a rate
$\valuerate(\statevec, \controlvec)$.
%
This problem is closely related to the multi-armed bandit problem, in that
the control (engagement) of existing activities is resource constrained.
%
However, the novelty of our framework is that activities may also create or refine new activities over time.
%
(In existing work, the acquisition of new activities is not a function of control.)

\subsection{Simplifying assumptions}

In the sequel we make some simplifying assumptions arising out of practical modeling.
For example, the business value of an activity is generally simply most sensitive to time---%
discounted as newer, better techniques/technologies are developed exogenously---%
and not to the condition ("hygiene") of the resources or procedures that implement that activity.
In contrast those conditions directly effect the resource requirements, e.g., in time and worker allotment, to engage the activity to produce.

Therefore, we can imagine a partition of the state space into $(\tau, \classvar)$, or
the simple age, and the current ``class'' of the activity.
The control in turn can be partitioned into the resource allotments as well as discrete \emph{activation} or mode.  Without loss of generality we can map activations onto classes.  

An activation could be characterized by:

\makecommand{\ratevar}{\lambda}
\makecommand{\actv}{a}
\makecommand{\resource}{r}
\makecommand{\agevar}{\tau}

$\ratevar^V_\actv$ --- starting value rate

$\beta^V_\actv$ --- value decay rate

$\ratevar^\classvar_\actv$ --- creation rate

$\ratevar^{\classvar \to \classvar'}_\actv$ --- refinement rate

$\Omega_{\actv \resource}$ --- activation per-resource ``impedence''; could be time-varying, for example, as technology becomes at first easier, and after harder to exploit as it ages

$\varphi_\actv$ --- activation function,
mapping resource allocation to activation engagement or \emph{effect}.

Here the density function of engagement of activation $\actv$ is
\[
x_\actv(\agevar) = \varphi_\actv\left(
	\Omega_{\actv \resource^1}^{-1}(\agevar) x_{\actv \resource^1}(\agevar),
	\Omega_{\actv \resource^2}^{-1}(\agevar) x_{\actv \resource^2}(\agevar),
	\ldots
\right).
\]

The rate of value generation (which we seek to maximize in steady-state equilibrium) is then given by
\[
\dot\valuevar_{\actv}
	= \int_{\agevar} 
		x_\actv(\agevar) \
		\ratevar^\valuevar_\actv
		e^{-\agevar \beta^\valuevar_\actv}
%		\left(\beta^\valuevar_\actv}\right)^\agevar
		\, d\tau
\]

In equilibrium,
we should have an activity age distribution per activity class of
\[
p_{\classvar}(t + \Delta t)[k]
	= p_{\classvar}(t)[k]
	+ \Delta t \ \sum_\actv
		x_\actv[k] 		
		\left[
			\ratevar_\actv^\classvar
			\ {\bf 1}_{k=0}
		+ \sum_{\classvar'} \ratevar_\actv^{\classvar' \to \classvar}
		- \sum_{\classvar'} \ratevar_\actv^{\classvar \to \classvar'}
		\right],
\]
where $k$ denotes the $k$th (age) sliver of width $\Delta t$.



\section{Event-based model}

An activation is the pair of a \emph{resource commitment} with an \emph{activation effect}.
A resource commitment is the tuple of a histogram $\rhist$ over resource classes and a commitment duration $\timevar$, to wit $(\rhist, \tau)$.




\section{Discrete-time model}

\section{A continuous model}

\subsection{Simple continuous productivity model}

\makecommand{\genEffort}{x}
\makecommand{\opprate}{\lambda}
\makecommand{\oppvalue}{v}
\makecommand{\oppdecay}{\beta}

\makecommand{\useEffort}{y}
\makecommand{\horizon}{\tau}

The control in this problem is division of a single unit of effort.
An effort $\genEffort$ generates new \emph{opportunities} at a continuous rate of $\opprate(\genEffort)$,
often simply $\opprate\genEffort$.
A unit of opportunity initially has value $\oppvalue_0$, but
that value decays exponentially over time, such that
a unit of opportunity age $t$ has value $\oppvalue_0 e^{-\oppdecay t}$
for $\oppdecay > 0$.
%
The remaining effort $\useEffort$ is spent exploiting existing opportunities.
Generally, the amount of opportunity exploited per unit time is an increasing, often continuous and perhaps linear function of the effort.
The most value is derived by spending effort on the freshest (least decayed) opportunities,
therefore the reward rate can be determined in terms of a ``horizon'' function  $\horizon(\genEffort,\useEffort)$, specifying the maximum age of an active opportunity.
In the doubly-linear case (i.e., production and exploitation both linear in effort),
we have simply $\horizon(\genEffort,\useEffort) = \horizon \useEffort / \genEffort$.
This leads ultimately to the reward function
%
\begin{equation}
R = \int_{t=0}^{\horizon(\genEffort,\useEffort)}
	\opprate(\genEffort) \, \oppvalue_0 e^{-\oppdecay t} \, dt,
\end{equation}
%
or often simply
%
%
\begin{equation}
R = \int_{t=0}^{\horizon \useEffort / \genEffort}
	\opprate \, \genEffort \, \oppvalue_0 e^{-\oppdecay t} \, dt
\end{equation}
%
or
\begin{equation}
R = \frac{\opprate \oppvalue_0}{\oppdecay} \genEffort
\left[
	1 - \left( e^{-\oppdecay \horizon} \right)^{\useEffort / \genEffort }
\right].
\end{equation}

Note that the factor $\oppdecay \horizon$ determines the optimal distribution of a unit of effort.


















\end{document}