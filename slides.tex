\documentclass[12pt]{beamer}
\usetheme{Rochester}
\usepackage[utf8]{inputenc}
\usepackage{amsmath}
\usepackage{amsfonts}
\usepackage{amssymb}
\usepackage{graphicx}
\usepackage{makecmds}
\usepackage{movie15}


\DeclareGraphicsExtensions{.pdf,.png,.jpg}

\author{Kyle Treleaven}
\title{State-space, Equilibria, and Stability and Lyapunov Methods I}
%\setbeamercovered{transparent} 
%\setbeamertemplate{navigation symbols}{} 
%\logo{} 
%\institute{} 
%\date{} 
%\subject{} 

\usefonttheme[onlymath]{serif}

% typesetting
\makecommand{\vitem}{\vfill\item}

% general math
\makecommand{\reals}{{\mathbb R}}
\makecommand{\timederiv}{\frac{d}{dt}}
\makecommand{\onehalf}{\frac{1}{2}}
\makecommand{\xpose}{{\rm T}}
\makecommand{\vecnorm}[1]{\left\|#1\right\|}

% specific math
\makecommand{\zerovec}{{\bf 0}}

\makecommand{\vecu}{{\bf u}}
\makecommand{\vecx}{{\bf x}}
\makecommand{\matA}{{\bf A}}
\makecommand{\matC}{{\bf C}}
\makecommand{\vectheta}{{\boldsymbol\theta}}

\begin{document}

\begin{frame}
\titlepage

\end{frame}

%\begin{frame}
%\tableofcontents
%\end{frame}


\begin{frame}
\frametitle{State-space Philosophy}

(1. Draw many possible trajectories of a bouncing ball)

\begin{block}{Problem}
Many possible trajectories from a particular location, of a bouncing ball under gravity.
\end{block}


\end{frame}


\begin{frame}

\begin{itemize}

\vitem Clearly, position is not all the information that is needed to determine trajectory.

\vitem But clearly, the ball couldn't just do anything, and only certain kinds of trajectories are possible.

\end{itemize}

\end{frame}



\begin{frame}
\begin{itemize}
\vitem Associating a ``velocity'' with the ball gives enough information to predict the trajectory exactly.

\vitem So the combination of position and velocity is some essential unit information we call state.

\vitem In other words, knowing the state at a particular moment in time is sufficient to simulate the entire trajectory.

\vitem Every possible trajectory can be recovered using some \emph{initial condition} $x_0$.
\end{itemize}

(put the Wikipedia quote?)

\vfill\null

\end{frame}




\begin{frame}

(2. Draw trajectories originating from initial conditions in a state space.)

\end{frame}




\begin{frame}
\frametitle{A formalism for real-valued signals}

\begin{itemize}
\vitem We begin observing a signal, or ``trajectory'',
\[
y(t) \in \reals.
\]

\vitem We call the present moment $t=0$, and are interested in $y(t)$ over $t \geq 0$ (future).

%\vitem Many trajectories are possible, but there are ``dynamical constraints''.

\vitem Substitute another signal $\vecx(t) \in \reals^d$ (perhaps more complicated), and functions $f : \reals^d \to \reals^d$ and $g : \reals^d \to \reals$, so that:

\begin{enumerate}
\vitem The system
\[
\begin{cases}
	\dot x = f(x) \\
	x(0) = x_0
\end{cases}
\]
has a unique solution over $t \geq 0$;

\vitem $y(t) = g( x(t) )$ over $t \geq 0$;

%In other words, applying $g$ to the state-trajectory recovers the desired signal.

\end{enumerate}

\end{itemize}

\vfill\null
\end{frame}



\begin{frame}
\frametitle{Bouncing Ball State-space Model}

Continuous dynamics: $\ddot x = -g$

``Jump'' dynamics: Whenever $x = 0$ and $\dot x < 0$, then
$x^+ = x$, $\dot x^+ = -\gamma \dot x$.

(Put solution equations?)

(3. Could annotate a trajectory in state-time space)



\end{frame}



\begin{frame}
\frametitle{Equilibria}

\begin{itemize}

\vitem \emph{Equilibria} are conditions such that the system ``stays put''.

\vitem Given a state-space $X$, the equilibria are 
\[
\left\{
	\vecx : \ f(\vecx) = \zerovec \implies \dot \vecx = \zerovec
\right\}
\]

\vitem In our example:
\[
f(\vecx) = \zerovec \implies
\begin{cases}
	\dot u = 0	\\
	\ddot u = -(k/m) u - (c/m) \dot u = 0
\end{cases}
\]

\vitem If $k > 0$, equilibrium of $u = 0$ and $\dot u = 0$ is \emph{unique}.

\vitem If $k = 0$, all $u$ are equilibria when $\dot u = 0$.

\end{itemize}

\end{frame}



\begin{frame}
Equilibrium for the bouncing ball?
\end{frame}



\begin{frame}
\frametitle{Ideal Pendulum System}

\begin{columns}

\column{.45\linewidth}

Pendulum trajectories satisfy

$\ddot\theta + (g/l) \sin \theta = 0$.

Use angle and angular velocity as a state-space.

$\vectheta = (\theta, \dot \theta)^\xpose$

$f(\vectheta)
	= \begin{pmatrix}
		\dot\theta \\
		-\frac{g}{l} \sin\theta
	\end{pmatrix}$.
	
$g > 0$ (thanks, universe), so:

\[
\begin{cases}
\dot\theta = 0 \\
\sin \theta = 0
\end{cases}
\]

\column{.45\linewidth}

\begin{center}
\includegraphics[width=\linewidth]{pic/Simple_gravity_pendulum.png}
\end{center}

\end{columns}

\end{frame}


\begin{frame}
\frametitle{Equilibria of the Pendulum}

Two [physically] distinct equilibria:

\begin{columns}

\column{.45\linewidth}

\begin{center}
\includegraphics[width=.4\linewidth]{pic/Pendulum_0deg.png}

$\dot\theta = 0$, $\theta=0$

\end{center}


\column{.45\linewidth}

\begin{center}
\includegraphics[width=.4\linewidth]{pic/Pendulum_180deg.png}

$\dot\theta = 0$, $\theta=\pi$
\end{center}


\end{columns}

\vfill
But we have a gut feeling that one of these equilibria is not ``stable''.

\end{frame}



\begin{frame}
\frametitle{Stability}

\begin{itemize}

\vitem Stability is a property of \emph{equilibria}, \emph{not} systems!

\vitem Stability of ``systems'' is a colloquialism which arose because of the importance of \emph{linear} systems;

\vitem Linear systems tend to have unique equilibrium (at origin).

\vitem But, it doesn't make any sense to ask \emph{``Is the pendulum stable?''}

\end{itemize}

\end{frame}



\begin{frame}
\frametitle{Lyapunov Stability}
\vfill
\begin{block}{Informal}
``If I put it there, and then I go make a sandwich, will it be like that when I get back?''
---Kyle
\end{block}

\vfill
\begin{block}{Formal}
An equilibrium $\bar\vecx$ is said to be \emph{stable in the sense of Lyapunov} if:
for every $\epsilon > 0$, there is $\delta >0$, so that
if $\vecnorm{\vecx(0)-\bar\vecx} < \delta$, then
$\vecnorm{\vecx(t) - \bar\vecx} < \epsilon$. 
\end{block}

\vfill\null

\end{frame}





\begin{frame}
\frametitle{Break-down}

\begin{block}{Formal}
$\bar\vecx$ is \textbf{stable in the sense of Lyapunov} if
\emph{for every $\epsilon > 0$}, there is $\delta >0$, so that
$\vecnorm{\vecx(0)-\bar\vecx} < \delta
 \implies \vecnorm{\vecx(t) - \bar\vecx} < \epsilon$. 
\end{block}

\begin{tabular}{rp{.6\linewidth}}
$\vecnorm{\vecx(0)-\bar\vecx}$
	& Initial distance from the equilibrium state \\ \\
	
$\vecnorm{\vecx(t)-\bar\vecx}$
	& Distance from the equilibrium state at time $t$ \\ \\
	
$\vecnorm{\vecx(0)-\bar\vecx} < \delta$
	& Start close; then... \\ \\

$\implies \vecnorm{\vecx(t)-\bar\vecx} \leq \epsilon$
	& Stay close (for all $t \geq 0$). \\

\end{tabular}

\vfill\null

\end{frame}





\begin{frame}
\frametitle{Other Properties of Equilibria}

\begin{block}{Attractive}
If there is $\delta > 0$ so that if $\vecnorm{\vecx(0)-\bar\vecx} < \delta$, then
$\vecnorm{\vecx(t)-\bar{\vecx}} \to 0^+$.
\end{block}

\begin{block}{\emph{Asymptotically} Stable}
Stable in the sense of Lyapunov \emph{and} attractive.
\end{block}

\begin{block}{Exponentially Stable}
If there are $\alpha, \beta, \delta >0$, so that if
$\vecnorm{\vecx(0)-\bar{\vecx}} < \delta$, then
$\vecnorm{\vecx(t)-\bar\vecx}
	\leq \alpha \vecnorm{\vecx(0)-\bar\vecx} e^{-\beta t}$
\end{block}

\end{frame}




\begin{frame}
\frametitle{Pathology?}
\end{frame}



\begin{frame}
\frametitle{Lyapunov's Second (``Direct'') Method}

\begin{itemize}
\vitem
It can be challenging, tedious, or impossible
to prove such claims about ``all trajectories'';
even to \emph{find} them (pendulum!).

\vitem
Lyapunov's ``direct'' method can provide a short-cut.

\end{itemize}

\vfill\null
\end{frame}




\begin{frame}
\frametitle{Lyapunov candidates}

\begin{itemize}

\vitem
Let $V : \reals^d \to \reals$ be a \emph{continuous}, scalar function.

\vitem
$V$ is a Lyapunov candidate if is \emph{positive-definite}, i.e.,

\[
\begin{cases}
	V(\zerovec) = 0 &\\
	V(\vecx) > 0 & \text{for all $\vecx \neq \zerovec$}.
\end{cases}
\]

\vitem
May be positive definite \emph{locally}; i.e., in a ``neighborhood''
(open set, containing $\zerovec$)

\vitem
(Assuming \emph{without loss of generality} that $\bar{\vecx} = \zerovec$.)

\end{itemize}
\vfill\null
\end{frame}




\begin{frame}

5. Define and demonstrate continuous, [positive/negative] [semi-] definiteness.

6. Define and demonstrate sub-level sets.

\end{frame}





\begin{frame}
\frametitle{Other types of stability}

\begin{block}{Types of Stability}
Asymptotic,
Exponential
\end{block}

Remark: There is an ``instability'' theory, but here only focus on proving an equilbrium \emph{is} stable.


\end{frame}




\begin{frame}
\frametitle{Pendulum Example}

(put the pendulum energy picture)

\[
E = mgh + \onehalf m v^2.
\]

$h = l(1-\cos\theta)$

$v = (l\dot\theta)^2$

\[
E
%ml \, g(1-\cos\theta) + \onehalf ml^2 \, \dot\theta^2
	= ml \left( g \, (1-\cos\theta) + \onehalf l \, \dot\theta^2 \right)
\]

\[
E' = g \, (1-\cos\theta) + \onehalf l \, \dot\theta
\]

(plot the surface, show positive definite!)

\end{frame}



\begin{frame}
\frametitle{Pendulum Example}
\[
\frac{\partial}{\partial\theta} E' = g \sin \theta
\]

\[
\frac{\partial}{\partial\dot\theta} E' = l \dot\theta
\]

\[
\dot E' =
	\frac{\partial}{\partial\theta} E' 
		\times f_{\theta}(\vectheta)
	+ \frac{\partial}{\partial\dot\theta} E'
		\times f_{\dot\theta}(\vectheta)
\]


\[
\dot E' =
	\left( g \sin\theta \times \dot\theta \right)
	+ \left( l \dot\theta \times (-g/l) \sin \theta \right) = 0
\]

(This is something we expected, being familiar with physics and conservation of total energy.)

Proof that the bottom equilibrium is \emph{marginally} stable i.s.L.

\end{frame}










\begin{frame}
\frametitle{Review of Last Time: Second-order State-space Model}

\begin{center}
\includegraphics[width=.3\linewidth]{pic/spring-mass-damper.png}
\end{center}

\begin{block}{Second-order, Linear Dynamical System}
$u(t)$ describes the position of a mass over time.

Known to obey the law:
$m \ddot u = -k u - c \dot u
	\implies\qquad
	\ddot u = -(k/m) u - (c/m) \dot u$,
	
where $\dot u = \timederiv u(t)$ (velocity), $\ddot u = \frac{d^2}{dt^2} u(t)$ (acceleration).
\end{block}

\end{frame}






\begin{frame}
\begin{block}{State-space Model}
Write
$\vecu(t) = \left( u(t), \dot u(t) \right)^\xpose$;
also written, 
$\vecu = \begin{pmatrix} u \\ \dot u \end{pmatrix}$.

(Element-wise derivative)

$\dot\vecu
	\doteq \timederiv \vecu(t)
	= ( \timederiv u(t), \timederiv \dot u(t) )^\xpose
	= \begin{pmatrix}
		\dot u \\
		\ddot u
	\end{pmatrix}$.

\end{block}

\begin{block}{SMD Model}
$\dot \vecu
	= \begin{pmatrix} \dot u \\ \ddot u \end{pmatrix}
	= \begin{pmatrix}
		\dot u \\
		-(k/m) u - (c/m) \dot u
		\end{pmatrix}$
		
$\dot \vecu
	= \begin{pmatrix} \dot u \\ \ddot u \end{pmatrix}
	= \begin{pmatrix}
		\dot u \\
		-(k/m) u - (c/m) \dot u
		\end{pmatrix}
	= \begin{pmatrix} 0 & 1 \\ -\frac{k}{m} & -\frac{c}{m} \end{pmatrix}
	\vecu$.
	
\end{block}

\end{frame}


\begin{frame}

\begin{itemize}

\vitem
$\matA := \begin{pmatrix} 0 & 1 \\ -\frac{k}{m} & -\frac{c}{m} \end{pmatrix}$

\vitem
Then $\dot\vecu$ is given by $f(\vecu) = \matA \vecu$

\vitem $\matC := \begin{pmatrix} 1 & 0 \end{pmatrix}$

\vitem
Then $u(t)$ is given by
$g(\vecu) = \matC \vecu$.

\end{itemize}
\vfill\null

\end{frame}


\begin{frame}
\frametitle{Solution}

\begin{itemize}

\vitem Admits a unique state-space trajectory determined by initial conditions $\vecu(0) = \vecu_0$
\[
  \vecu(t) = \vecu_0 e^{t \matA }
  \qquad t \geq 0.
\]

\vitem Particle trajectory is given by $u(t) = \matC \vecu(t)$ for $t \geq 0$.

\end{itemize}
\vfill\null

\end{frame}




\begin{frame}
Put ``state-space'' locus plots of equilibria in the two cases.
\end{frame}





\begin{frame}
\frametitle{Stability for LTI Systems (Continuous)}


$\dot x = Ax$

The notation $P > 0$ means that matrix P is positive definite.
Given any $Q > 0$, there exists a unique $P > 0$ satisfying 
$A^\xpose P + P A + Q = 0$ if and only if the origin is the globally asymptotically stable equilibrium of the linear system $\dot x = A x$.

If and only if $\lambda \in eig(A) => Re(\lambda) < 0$

The quadratic function $V(z) = z^\xpose P z$ is a Lyapunov function that can verify stability.

\end{frame}



\begin{frame}
\frametitle{Pathologies}

Examples: globally attractive equilibrium, not stable, even i.s.L. (weakest sense)

also, "radial unboundedness" ("escape")

\end{frame}








\begin{frame}
\frametitle{Remarks}

There is an instability theory


Similar theories for discrete systems (e.g., computer programs)

\end{frame}









\begin{frame}
BACKUP
\end{frame}





\makecommand{\matJ}{{\bf J}}

\begin{frame}
\frametitle{Lyapunov's Indirect Method (Linearization)}

Given $\dot \vecx = f(\vecx)$ and equilibrium $\vecx_e$,
form system

$\tilde\vecx = \vecx - \vecx_e$

$\dot{\tilde\vecx}
	= \left.
		\frac{\partial f }{\partial\vecx} \right|_{\vecx=\vecx_e}
		\, \tilde\vecx$.

(Use this one against the pendulum inverted equilibrium?)

\end{frame}




\begin{frame}{Review of Second-order State-space Model (from last time)}

\begin{equation}
V = b \left( y - x^{2}\right)
	\left(
		-\onehalf a + \frac{b}{2c} \left(k + m\right) \left(- x^{2}{\left (t \right )} + y{\left (t \right )}\right) + 0.5 x{\left (t \right )}\right)
		+ \left(a - x{\left (t \right )}\right) \left(- 0.5 b \left(- x^{2}{\left (t \right )} + y{\left (t \right )}\right) + \frac{0.5}{c k} \left(a - x{\left (t \right )}\right) \left(c^{2} + k m + m^{2}\right)\right)
\end{equation}

\begin{equation}
\dot V = -b^2 \left( y - x^2 \right) - \left( a - x \right)^2
\end{equation}

\end{frame}













\begin{frame}\end{frame}
\begin{frame}\end{frame}
\begin{frame}\end{frame}
\begin{frame}\end{frame}





\end{document}